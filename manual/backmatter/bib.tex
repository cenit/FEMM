\begin{thebibliography}{99}
\bibitem{Plonus} M. Plonus, {\em Applied electromagnetics}. McGraw-Hill, 1978.
\bibitem{Hoole} S. R. Hoole, {\em Computer-aided analysis and design of
electromagnetic devices}, Elsevier, 1989.
\bibitem{Jackson} J. D. Jackson, {\em Classical electrodynamics, $2^{nd}$ ed},
Wiley, 1975.
\bibitem{FrankWhite} F. M. White, {\em Heat and mass transfer}, Addison-Wesley, 1988.
\bibitem{Haberman} R. Haberman, {\em Elementary applied partial differential equations}, Prentice-Hall, 1987.
\bibitem{stoll} R. L. Stoll, {\em The analysis of eddy currents},
Oxford University Press, 1974.
\bibitem{mcfee} S. McFee, J. P. Webb, and D. A. Lowther, ``A
tunable volume integration formulation for force calculation in
finite-element based computational magnetostatics,'' IEEE
Transactions on Magnetics, 24(1):439-442, January 1988.
\bibitem{henforce} F. Henrotte, G. Deliege, and K. Hameyer, ``The
eggshell method for the computation of electromagnetic forces on
rigid bodies in 2D and 3D,'' CEFC 2002, Perugia, Italy, April
16-18, 2002.
\href{http://www.esat.kuleuven.ac.be/electa/publications/fulltexts/pub_942.pdf}{(pdf version)}
\bibitem{luaman} R. Ierusalimschy, L. H. de Figueiredo, and W. Celes,
{\em Reference Manual of the Programming Language Lua 4.0}
\href{http://www.lua.org/manual/4.0/}{\tt http://www.lua.org/manual/4.0/}
\bibitem{allaire} P. E. Allaire, {\em Basics of the finite element
method}, 1985.
\bibitem{silv} P. P. Silvester, {\em Finite elements for electrical engineers},
Cambridge University Press, 1990.
\bibitem{Henrotte} F. Henrotte {\em et al}, ``A new method for
axisymmetric linear and nonlinear problems,'' {\em IEEE
Transactions on Magnetics}, MAG-29(2):1352-1355, March 1993.
\bibitem{fletcher} C. A. Fletcher, {\em Computational techniques
for fluid dynamics}, Springer-Verlag, 1988.
\bibitem{freund} R. W. Freund, ``Conjugate gradient-type methods
for linear systems with complex symmetric coefficient matrices,''
SIAM Journal of Scientific and Statistical Computing,
13(1):425-448, January 1992.
\bibitem{Daze} E. F. D'Azevedo, P. A. Forsyth, and W. Tang,
``Ordering methods for preconditioned conjugate gradient methods
applied to unstructured grid problems,'' SIAM J. Matrix Anal.
Appl., 12(4), July 1992.
\bibitem{zen} O. C. Zienkiewicz and J. Z. Zhu, `` The
superconvergent patch recovery and {\em a posteriori} estimates,
part 1:  the recovery technique,'' International Journal for
Numerical Methods in Engineering, 33:1331-1364, 1992.
\bibitem{chen} Q. Chen and A. Konrad, ``A review of finite element
open boundary techniques for static and quasistatic electromagnetic
field problems,'' {\em IEEE Transactions on Magnetics},
33(1):663-676, January 1997.
\bibitem{LowFree} E. M. Freeman and D. A. Lowther, ``A novel
mapping technique for open boundary finite element solutions to
Poissons equation,'' {\em IEEE Transactions on Magnetics},
24(6):2934-2936, November 1988.
\bibitem{LowFreeFor} D. A. Lowther, E. M. Freeman, and B.
Forghani, ``A sparse matrix open boundary method for finite element
analysis,'' {\em IEEE Transactions on Magnetics}, 25(4)2810-2812,
July 1989.
\bibitem{LowFreeAxi} E. M. Freeman and D. A. Lowther, ``An open
boundary technique for axisymmetric and three dimensional magnetic
and electric field problems,'' {\em IEEE Transactions on
Magnetics}, 25(5):4135-4137, September 1989.
\bibitem{jackmecrow} A. G. Jack and B. C. Mecrow, "Methods for magnetically nonlinear
problems involving significant hysteresis and eddy currents," IEEE
Transactions on Magnetics, 26(2):424-429, March 1990.
\bibitem{okelly} D. O'Kelly, ``Hysteresis and eddy current losses in
steel plates with nonlinear magnetization characteristics,''
Proceedings of the IEE, 119(11):1675-1676, November 1972.
\end{thebibliography}